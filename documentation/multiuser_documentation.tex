\documentclass[a4paper, titlepage]{article}

\usepackage[ngerman]{babel}
\usepackage[utf8]{inputenc}
\usepackage[T1]{fontenc}
\usepackage{graphicx}

\title{Multiuser Applikation}
\author{Damien Flury}
\date{04. November 2019}

\begin{document}
    \maketitle

    \tableofcontents
    \newpage

    \section{Anwendungsfälle}
    Ein Anwendungsfalldiagramm finden Sie in Abbildung \ref{usecase}.

    \begin{figure}
        \includegraphics[width=\textwidth]{images/UseCaseDiagram.png}
        \caption{Use Case-Diagramm}
        \label{usecase}
    \end{figure}
    \subsection{Akteure}
    Benutzer können sich anmelden, abmelden, und Entries managen.

    \subsection{Anforderungen}
    \subsubsection{Create Account}
    Neue Benutzer können einen eigenen Account erstellen. Dazu
    brauchen sie eine Email-Adresse und ein Passwort, welches
    den Anforderungen entsprechen.

    \subsubsection{Login}
    Bestehende Benutzer können sich anmelden. Dazu werden wiederum
    die Email-Adresse, welche sie zur Erstellung verwendet wurde,
    und das dazugehörige Passwort benötigt.

    \subsubsection{Logout}
    Eingeloggte Benutzer können sich wieder abmelden. Dazu müssen
    sie zunächst angemeldet sein.

    \subsubsection{Manage Entries}
    Eingeloggte Benutzer können neue Entries erstellen, lesen,
    bearbeiten und wieder löschen.

    \subsection{Nicht funktionale Anforderungen}
    \subsubsection{Performance}
    Die Datenbankabfragen müssen möglichst performant ablaufen, um die
    Userexperience nicht einzuschränken.
    \subsubsection{Design}
    Einheitliches Design in der Webapplikation, um die Userexperience zu
    optimieren.
    \subsubsection{Simple Authentifizierung}
    Die Applikation benötigt Authentifizierung. Die Tokens werden im
    Frontend gespeichert, sodass der Benutzer sich nicht jedesmal erneut
    einloggen muss. Für das Login werden lediglich Email und Passwort
    benötigt.
    \subsubsection{Sicherheit}
    Um die bestmögliche Sicherheit zu garantieren, wird HTTPS verwendet
    und ein ORM verwendet, um Database Injection zu vermeiden.

    \section{Datenhaltung}
    Die Applikation besteht aus vier Datenklassen (siehe Abbildung \ref{fachklassen}):
    \begin{itemize}
        \item ApplicationUser
        \item Entry
        \item Position
        \item Department
    \end{itemize}

    \begin{figure}
        \includegraphics[width=\textwidth]{images/Fachklassendiagramm.png}
        \caption{Fachklassendiagram}
        \label{fachklassen}
    \end{figure}

    \section{Architektur}
    \subsection{Packagediagramm}
    Ich verwende sechs Namespaces, zwei für die Datenbank und vier für die
    GraphQL API (siehe Abbildung \ref{packages}).

    \begin{figure}
        \includegraphics[width=\textwidth]{images/Packagediagramm.png}
        \caption{Packagediagramm}
        \label{packages}
    \end{figure}
    
    \subsection{Klassendiagramm}
    Ich verwende Klassen für die Datenbank-Entitäten und für die
    GraphQL Types. Ausserdem gibt es eine Startup-Klasse für die 
    Projektkonfiguration (siehe Abbildung \ref{classdiagram}).

    \begin{figure}
        \includegraphics[width=\textwidth]{images/Klassendiagramm.png}
        \caption{Klassendiagramm}
        \label{classdiagram}
    \end{figure}

\end{document}